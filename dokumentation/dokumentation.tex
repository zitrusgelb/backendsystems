\documentclass[a4paper, 11pt]{article}
\usepackage[utf8]{inputenc}
\usepackage{graphicx}
\renewcommand{\familydefault}{\sfdefault}

\begin{document}

    \title{\vspace{-3cm}\includegraphics[width=0.5\textwidth]{thws.png}\\ Project documentation for the portfolio exam}
    \date{\today}
    \author{Tim Knüttel (5123101) \and Adrian Stelter (5123031) \and Laurin Neubert (6824016) \and Jakob Hämmelmann (5123113)}



    \maketitle
% 4-6 Pages


    \section{Introduction of the Project Topic}\label{sec:introduction-of-the-project-topic}

    \subsection{Domain}\label{subsec:domain}
    The domain of our project is a social media platform like X (formerly Twitter), Threads or Bluesky.
    Our goal for the project is to create a platform that enables users to interact with one another through means of text posts.
    The main objective here is to give users the opportunity to interact not only by posting their own posts, but also to give feedback through likes and comments.
    Tags help the users to find posts based on interests or topics.

    \subsection{Use Cases}\label{subsec:use-cases}

    \subsubsection{Posts}
    The central and most important use case for our project is the ability for users to post text posts. When posting, users should be able to fill their post with content, also it should be visible which user posted at what exact time.

    \subsubsection{Replies \& Likes}
    Users should be able to interact with posts directly by replying to and \("\)liking\("\) them. This gives users the ability to converse on the platform and discuss their favorite topics. Additionally, replying keeps all the different posts together, so there can be an easy understanding of which post is a new topic and which post was written in response to an already existing topic.

    \subsubsection{Tags}
    Tags can be used to link a topic to the created post. By searching for posts tagged with the desired topic, users can easily find posts that are relevant to what they wish to converse about. This is designed to increase the usability and also user retention.

%Summarize the domain and describe a few (but not all) use cases of your software.

%Clear and concise summary of the domain.
%Description of use cases is relevant and demonstrates an understanding of the problem domain.


    \section{Description of the Software Architecture}\label{sec:description-of-the-software-architecture}
%Explain your interpretation of the hexagonal architecture.
%What belongs to the domain?
%What is the business logic?
%What are the infrastructure components?
%How did you implement ports and adapters?
%What was the most difficult aspect to learn when implementing this architecture?

%Correct and thorough explanation of the hexagonal architecture.
%Clear distinction between domain, business logic, and infrastructure components.
%Explanation of ports and adapters and how they were implemented.
%Reflection on challenges faced during implementation.


\section{Erklärung der API-Technologie} \label{sec:erklaerung-der-api-technologie}

Für dieses Projekt haben wir uns für eine \textbf{RESTful API-Architektur mit Spring Boot} entschieden. Diese Wahl basiert auf mehreren Faktoren. REST ist ein weit verbreiteter Industriestandard, der die Integration mit anderen Systemen erheblich erleichtert. Zudem ermöglicht die \textit{stateless} Kommunikation eine hohe Skalierbarkeit, was insbesondere für groß angelegte Anwendungen von Vorteil ist. Darüber hinaus bietet REST eine hohe Flexibilität, da es mit verschiedenen Frontend-Technologien und mobilen Anwendungen kompatibel ist. Ein weiterer entscheidender Punkt ist die einfache Implementierung: Spring Boot stellt eine umfassende Unterstützung für REST-Controller, Authentifizierung und Persistenz bereit, was die Entwicklung effizienter macht.

Diese Technologie bringt für unser Projekt eine Reihe von Vorteilen mit sich. REST ist \textbf{einfach und leichtgewichtig}, da es auf JSON über HTTP basiert. Dies erleichtert sowohl die Implementierung als auch das Debugging. Dank der \textit{stateless} Architektur wird zudem eine effiziente Nutzung von Ressourcen ermöglicht. Ein weiterer wichtiger Aspekt ist die \textbf{Sicherheit}: Unsere API nutzt JWT-Authentifizierung, um API-Aufrufe abzusichern. Zudem lässt sich REST hervorragend mit Spring Data JPA kombinieren, wodurch eine nahtlose Integration mit der Datenbank gewährleistet ist.

Trotz dieser Vorteile gibt es einige Einschränkungen und Kompromisse, die berücksichtigt werden müssen. Eine der größten Herausforderungen ist der \textbf{Mangel an Echtzeitkommunikation}. Im Gegensatz zu WebSockets oder GraphQL-Subscriptions bietet REST keine integrierte Unterstützung für Echtzeit-Updates. Außerdem besteht die Möglichkeit des \textbf{Over-Fetching oder Under-Fetching}, da Clients unter Umständen mehr oder weniger Daten als erforderlich erhalten, was die Performance beeinträchtigen kann – insbesondere bei großen Datenmengen. Ein weiteres Problem ist das \textbf{State-Management}: Da REST \textit{stateless} ist, kann die Abwicklung komplexer Transaktionen über mehrere API-Aufrufe hinweg herausfordernd sein.

Trotz dieser Limitationen überwiegen in unserem Anwendungsfall die Vorteile der RESTful API mit Spring Boot, wodurch eine robuste, flexible und gut skalierbare Lösung für unser Projekt gewährleistet wird.
%Why did you choose this technology?
%Do you see any advantages of this technology for this project?

%Justification for choosing the specific API technology (gRPC, GraphQL, or REST).
%Identification of advantages relevant to the project.
%Consideration o/f any limitations or trade-offs of the chosen technology.


    \section{Implementation Details}\label{sec:implementation-details}
%For example, how did you implement authentication?
%How did you implement the mapping in the adapters?
%Did you use any framework like REST Easy, Jersey, or Spring?
%Why did you choose this framework?
As a framework we choose Quarkus and therefore REST Easy. Already collecting experience with Quarkus is a benefit for the members of our project. Furthermore, from a future proofing perspective Quarkus is ready for the use as a base for micro services. Meaning the application being split into multiple small parts.\\
To implement mapping we went with MapStruct. We choose it because it is a modern approach to mapping as well as being generated instead of manually written, this reduces error in implementation.
For authentication we choose to implement a custom middleware that interacts with the THWS Authentication System and takes the Bearer token it generates. This decision relies on the ever more increasing popularity of using bigger providers like Google or GitHub as the sole authentication source.

%Explanation of implementation choices (e.g., authentication, adapter mapping).
%Justification of framework choice (e.g., REST Easy, Jersey, Spring).
%Level of detail in the technical description (must be clear and understandable).


    \section{Testing Strategy}\label{sec:testing-strategy}
%How did you implement unit and integration tests?

%Explanation of the testing strategy, including unit and integration tests.
%Description of how tests were implemented and what was tested.
%Reflection on the effectiveness of the testing strategy.


    \section{Learning Outcomes and Reflection}\label{sec:learning-outcomes-and-reflection}
    Laurin's notes: next time, more time planning before we start coding (esp. when the API-technology is not yet picked); smaller goals that can be achieved quicker so we can support earlier in the process if someone needs help; utilising Git to it's full potential (namely creating milestones and issues;)

%What did you learn from this project?
%What worked well, and what would you do differently in the future?
%Reflect on team collaboration and the use of hexagonal architecture.

%Clear articulation of key learning outcomes from the project. (1 Point)
%Honest reflection on what worked well and what could be improved. (1 Point)
%Insightful discussion of team collaboration and the use of hexagonal architecture.

    \section*{Caption}
    This article was drafted and refined using GPT-4 based on an outline containing related information. The GPT-4 output was reviewed, revised, and enhanced with additional content. It was then edited for improved readability and active tense, partially using Grammarly.
\end{document}
